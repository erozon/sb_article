\documentclass[titlepage, hidelinks, 12pt]{article}


%for custom page numbering:
\usepackage{fancyhdr}

\usepackage{lipsum}
\usepackage{hyperref}
\usepackage{palatino}
\usepackage{tikz}
\usepackage{chngcntr}
\counterwithin{figure}{section}
%\usepackage{breqn} % useful for breaking equations across multiple lines automatically. Breaks everything.


\usepackage{setspace}
%\usepackage{indentfirst} %tex default is no indent on first paragraph after heading
\usepackage{url}
\usepackage{amsmath, amssymb, amsfonts, amsthm}
\usepackage{float}
\usepackage{subfig}
\usepackage{graphicx}
\usepackage{environ, enumerate}
%\usepackage{mathbbol}
%\DeclareSymbolFontAlphabet{\amsmathbb}{AMSb}
\graphicspath{ {images/} }
\providecommand{\keywords}[1]{\textbf{\textit{Keywords---}} #1} 
\usepackage[format=plain,
            labelfont={bf, it},
            textfont=it]{caption}





%%%%%%%%%
% indentation
%%%%%%%%%

\setlength\parindent{0pt}
\setlength{\parskip}{\baselineskip}

\setlength{\voffset}{-1cm}
\setlength{\textwidth}{17cm}
\addtolength{\textheight}{2cm}
\setlength{\footskip}{1cm}
\addtolength{\oddsidemargin}{-2cm}
\addtolength{\evensidemargin}{-2cm}

\widowpenalty10000
\clubpenalty10000



%%%%%%%%%
% page numbering and logo
%%%%%%%%%

\pagestyle{fancy}            

\fancyhead[L]{}
\fancyhead[C, R]{}
\fancyfoot[L]{}
\fancyfoot[C]{\thepage}
\fancyfoot[R]{\includegraphics[scale=.35]{sb_logo_black.png}}
\renewcommand{\headrulewidth}{0pt}
\renewcommand{\footrulewidth}{0pt}


%Def, Lemma, Theorem, Corollary environment
\theoremstyle{plain}
\newtheorem{theorem}{Theorem}[section]
\newtheorem{corollary}[theorem]{Corollary}
\newtheorem{lemma}[theorem]{Lemma}
\newtheorem{proposition}[theorem]{Proposition}
\newtheorem{conjecture}[theorem]{Conjecture}
%\newtheorem*{proof}{Proof}
\theoremstyle{remark}
\newtheorem*{remark}{Remark}
\newtheorem*{example}{Example}
\theoremstyle{definition}
\newtheorem{definition}[theorem]{Definition}

%New commands
\newcommand{\Q}{\mathbb{Q}}
\newcommand{\Z}{\mathbb{Z}}
\newcommand{\N}{\mathbb{N}}
\newcommand{\R}{\mathbb{R}}
\newcommand{\T}{\mathcal{T}}
\newcommand{\E}{\mathbb{E}}
\newcommand{\betahat}{\hat{\beta}}


%New math operators
\DeclareMathOperator{\ringchar}{char}
\DeclareMathOperator{\diag}{diag}
\DeclareMathOperator{\disc}{disc}
\DeclareMathOperator*{\argmax}{arg\,max}
%\DeclareMathOperator{\exp}{exp}
\renewcommand\d[1]{\:\textrm{d}#1}
\newcommand*\diff{\mathop{\!\mathrm{d}}}
\doublespacing
\begin{document}
\begin{singlespace}
\begin{titlepage}
   \begin{center}
       \vspace*{1cm}
\Huge
       \textbf{Squirrel banking} \\
\Large
       \vspace{0.5cm}
       Evolutionarily optimal caching. 
      
            
       \vspace{4.1cm}
       \includegraphics[scale = 1.4]{sb_logo_black.png} \\
       \vspace*{\fill}
       \textbf{Eric Rozon} \\

            
       Department of Mathematics\\
       University of British Columbia
            
   \end{center}
\end{titlepage}





\begin{abstract}

    First read, last written. 
    \bigskip

    \begin{keywords}
    Some, key, words. 
    \end{keywords}

\end{abstract}

\newpage

\tableofcontents

\newpage

%\section*{Acknowledgements}
%\addcontentsline{toc}{section}{Acknowledgement}
%My most sincere thanks are owed to Monica Nevins, without the support and guidance of whom none
%of this project would have been possible. Monica, it was a joy and a privilege to work so
%closely with you this semester, and I hope you are aware that my interest in mathematical research
%is due in no small part to your enthusiasm. I would also like to thank Professor Rafal Kulik for
%allowing me the opportunity to present part of this project during the University of Ottawa's final
%math club meeting of 2018. Finally, if you are reading this report, then thank you. While the project
%was stimulating enough in its own right that I do not feel it must be read by others to have been
%worthwhile, it is also fun to think that someone else may find some joy or usefulness in my work.
%\newpage
%\input{texfiles/outline}
%\newpage
\end{singlespace}





\section{Introduction}
We are often told that patience is a virtue. At the same times, the early bird gets the worm. 
While the sentiments expressed are in direct opposition, common sense dictates that each paradigme is of value. 
When making a decision between immediate and delayed gratification, the reconciliation of patiece and hastiness
may be translated into a question of how one ought to value time versus material gain. 

The theory of delay discounting has strong roots in economic theory \cite{sozou98, fedus19}. Models of discounting seek to axiomatize rational
behaviour and prescribe optimal preferences (patience or hastiness) facing decisions across different time horizons \cite{mazur97, ainslie75, green81}. 
Nevertheless, popular theories of discounting have severe limitations. In particular, experiments involving both human and non-human animals 
consistently find behaviour which is both qualitatively and quantitatively different from that predicted by standard 
models \cite{maia09, green04, frederick02}.

Axiomatic approaches to understanding decisions between patience and impulsiveness are insufficient. Instead, in this article we consider
an evolutionary approach to understanding patience. The brain (both human and non-human) often makes decisions that we do not fully consider
or rationalize; such brain functions are governed by what has been coined the adaptive unconscious \cite{wegner02}. In an evolutionary time scale,
those brains which make quick decisions which are favourable for gene proliferation are selected for and become the dominant brain type in a population.
In particular, we apply this reasoning to patience: those individuals whose patience is most adaptive will fixate in a population and drive all other
types to extinction.

In this article, we put forward an idealized model of squirrels foraging and banking nuts. Squirrels may either bank their nuts to guard against
starvation, or else consume additional nuts to produce offpsring. Ecologically, the squirrels in our model face a survival versus reproduction
tradeoff. From the perspective of patience, this translates into a question of fewer offspring more immediately (smaller and sooner benefit) 
or reproduction over a longer time span resulting in more offspring but later in life (larger and later benefit). By considering squirrels with
varying levels of patience (as measured by cache size before reproduction), we address the question: How much patience is evolutionarily optimal?

\section{Squirrel Mechanics and Accounting}
Suppose a squirrel forrages daily for nuts. Each day, the squirrel finds either $0, 1,$ or $2$ nuts, with probabilities $p_0, p_1$, and $p_2$
respectively. If the squirrel consumes $0$ nuts, it dies. By consuming a single nut, the squirrel keeps itself alive. Finally, should the squirrel
consume $2$ nuts, it both keeps itself alive and produces an offspring. The offspring is subject to a one day period of parental care during
which it survives if and only if its parent survives. 

The life of a single squirrel is a stochastic process. The squirrel's survival and reproductive events are determined by the random process of
finding nuts. In a large (or infinite) population of squirrels, we assume nut findings are independent (both between individuals and day to day),
so that we may interpret that a fraction $p_n$ find $n$ nuts each day (for $n=0,1,2$). We give squirrels the capacity to save nuts, and consider
the restricted class of saving/consumption strategies indexed by $\beta\in\left\{ 0, 1, 2 \ldots \right\}$. A squirrel with strategy $\beta$
saves nuts until it has a cache size of $\beta$, after which it uses all nuts to reproduce. 


We first consider the case of squirrels who bank no nuts, immediately using all available nuts for reproduction. That is, we consider
first the case of squirrels with strategy $\beta = 0$. 
The introduction of a demographic matrix
helps to keep track of a population. Squirrels can be segregated into two distinct populations: they are either providing parental 
care or they are not. 
Transitions probabilites are linear and therefore we may keep track of a population using a demorgraphic matrix, wherein the $(i,j)$th entry is the
probability of transitioning from state $j$ to state $i$. See \cite{stearns92} for a more in depth discussion of such matrices. 
We let $D_0$ be the demographic matrix for squirrels banking no nuts.  
\begin{equation}
D_0 =
\begin{pmatrix}
    p_1 & 2(p_1 + p_2) \\ p_2 & 0
\end{pmatrix}
    \label{eqn:D0}
\end{equation}

Consider next a population of squirrels, each of which banks one nut before using additional resources to produce offspring: $\beta =1$. 
In this case, squirrels may be partitioned into three distinct categories:
\begin{enumerate}
    \item Squirrels having cached $0$ nuts;
    \item squirrels having cached $1$ nut and not providing parental care; and
    \item squirrels providing parental care.
\end{enumerate}
The population dynamics of  $\beta = 1$ squirrels are easily accounted for in the matrix $D_1$. Once more, the $(i,j)$th entry is the
probability of transitioning from state $j$ to state $i$.
\begin{equation}
D_1 =
\begin{pmatrix}
    p_1 & p_0 & 1+p_0 \\ p_2 & p_1 & 2(p_1 + p_2) \\ 0 & p_2 & 0
\end{pmatrix}
    \label{eqn:D1}
\end{equation}

More generally, we can construct demographic matrices for general strategies $\beta \ge 2$. 
A population with cache level $\beta$ can be divided into $\beta+2$ categories:
\begin{enumerate}
    \item Squirrels having cached $0$ nuts;
    \item squirrels having cached $1$ nut;
    \item []\dots
    \item[$\beta+1$.] squirrels having cached $\beta$ nuts and not providing parental care; and
    \item[$\beta+2$.] squirrels providing parental care. 
\end{enumerate}
A population of $\beta$ strategy squirrels is most easily accounted for using the demographic matrix $D_\beta$. 
\begin{equation}
D_\beta =
\begin{pmatrix}
    p_1 & p_0 & 0 & \cdots & 0 & 1 \\
    p_2 & p_1 & p_0 & \cdots & 0 & 0 \\
    0 & p_2 & p_1 & \cdots & 0 & 0 \\
    \vdots & \vdots & \vdots & \ddots & \vdots & \vdots \\
    0 & 0 & 0 & \cdots & p_0 & p_0 \\ 
    0 & 0 & 0 & \cdots & p_1 & p_1 + p_2 \\ 
    0 & 0 & 0 & \cdots & p_2 & 0 \\ 
\end{pmatrix}
    \label{eqn:Dbeta}
\end{equation}

\section{Evolutionary Dynamics and Fitness}
At a high level, we are trying to understand how adaptive pressures determine the patience of a population of squirrels. There are several
possible outcomes which we might observe.
\begin{itemize}
    \item Stability: adaptive pressures dictate that some $\beta^*$ is optimal. In the long run, only squirrls using strategy $\beta = \beta^*$ will
        thrive, and all others will die out.
    \item Cyclic outcomes: there is no one optimal strategy, but rather the population oscillates between various strategies indefinitely
        in a recurring patern.
    \item Non-cyclic dynamics: the population does not converge to any strategy, and also does not repeat its trajectory. Instead, adaptive
        pressures force squirrels to adopt different strategies indefinitely. 
\end{itemize}
To determine which outcome occurs, we need to describe in more depth how squirrels interact with their environment. As a first attempt,
suppose that the distribution of nuts $(p_0, p_1, p_2)$ is independent of the quantity of squirrels foraging. Fix a distribution
$(p_0, p_1, p_2)$. Absent density dependent
resources, squirrel populations will grow (or decay) exponentially indefinitely. A convenient feature of the demographic matrices $D_\beta$ (subject
to minor diagonalizability requirements, see \cite{stearns92} for a discussion),
is that the maximal eigenvalue of $D_\beta$ (call it $\lambda_\beta$) is the population growth rate. In populations that grow exponentially,
those with maximal growth rate dominate. Therefore, if resources are independent of population size, the adaptive outcome is to choose
$\beta$ so as to maximize $\lambda_\beta$. Mathematically, we seek to compute 
\begin{equation}
    \beta^*(p_0, p_1, p_2) := \argmax_{\beta\in\N} \lambda_\beta.
    \label{eqn:betastar}
\end{equation}
The problem described in equation \ref{eqn:betastar} is analytically intractible. We get some intuitive understanding of the nature of the problem
by numerical approximations. For instance, for $(p_0, p_1, p_2) = (0.3125, 0.3125, 0.375)$, we compute:
\begin{table}[H]
    \centering
\begin{tabular}{l|l}
$\beta$ & $\lambda_\beta$    \\ \hline
0       & 0.891123501019053  \\
1       & 1.0042807386908956 \\
2       & 1.0109688024833265 \\
3       & 1.0119993543876564 \\
4       & 1.0116325207441923 \\
5       & 1.010911192617546  \\
$\vdots$ & $\vdots$
\end{tabular}
\end{table}
The value of $\lambda_\beta$ increases until $\beta = 3$, after which it decreases indefinitely. More generally, we experimentally find that
for each distribution of nuts $(p_0, p_1, p_2)$, the value of $\lambda_\beta$ is maximized at some $\beta^*$. 
Our numerical work indicates that the optimal cache level $\beta^*$ increases as the distribution of nuts becomes less bountiful.
That is, squirrels foraging in an environment where nuts are plentiful have maximal fitness when they exhibit no patience and use all available
resources for reproduction. If nuts are less plentiful, however, squirrels are better served to be patient and delay reproduction in favour of
higher survival probability. A pleasing result of our model setup is that for every cache level $\beta$, there exists some distribution of nuts
$(p_0, p_1, p_2)$ for which $\beta^*(p_0, p_1, p_2) = \beta$.

\begin{figure}[H]
    \centering
    \includegraphics[scale = 0.7]{ternary_optimal_beta.png}
    \caption[Ternary plot showing $\lambda$ maximizing patience for each nut distribution $(p_0, p_1, p_1)$.]{The simplex of points $(p_0, p_1, p_2)$ such that $p_0 + p_1 + p_2 =1$ describes the space of all possible nut distributions.
        The simplex can be projected onto two dimensional space as a ternary diagram. Each point is a distribution of nuts $(p_0, p_1, p_2)$,
        and 
    the colouring of each point indicates the cache level resulting in maximal fitness. When nuts are plentiful (upper right of the diagram),
squirrels maximize fitness by banking $0$ nuts. If nuts are less available, meaning the distribution shifts toward the bottom left, 
then banking more nuts is favourable. As $p_2\to p_0$, $\beta^*\to\infty$. For visualization purposes, we stop at $\beta^* = 6$. 
If $p_0 > p_2$, meaning a squirrel is more likely to find $0$ nuts than $2$ nuts, then fitness $\lambda_\beta < 1$
for every finite $\beta$. Infinite patience is required to maintain constant population size.}
    \label{fig:optimal_beta}
\end{figure}






\section{Density Dependence and Trait Substitution Sequence}
The present setup assumes the distribution of nuts $(p_0, p_1, p_2)$ is constant in time. The results shown in Figure \ref{fig:optimal_beta}
are therfore the dominant patience phenotype for each constant distribution $(p_0, p_1, p_2).$ However, it is unreasonable to suppose that
exponential growth takes place indefinitely. 
A more likely alternative is that resources
are density dependent. Without specifying the mechanism via which population density determines the distribution of nuts, we abstractly
posit that some such mechanism exists and ensures that populations neither grow nor decay exponentially, and rather remain at a constant size.


Let us suppose that a population of $\beta = 0$ squirrels is in equilibrium. 
Ecological forces have driven the distribution of nuts to some $(p_0, p_1, p_2)$ at which $\lambda_0 =1$,
meaning the squirrel population is of constant size. Spontaneously, a mutant squirrel with cache level $\beta = 1$ is born. Computing numerically,
we see that for all distributions $(p_0,p_1,p_2)$ satisfying $\lambda_0(p_0, p_1, p_2) = 1$, $\lambda_1(p_0,p_1,p_2)>1$. That is, the fitness
of the mutant (patient) squirrel is greater than that of the resident population. It follows that the mutant population will grow, putting additional
pressure on resources. The mutant population therefore profits at the expense of the residents, until the $\beta = 0$ squirrels have all died off
and the $\beta = 1$ squirrels have driven nut distribution to some $(p_0', p_1', p_2')$ satisfying $\lambda_1(p_0', p_1', p_2') = 1$. 

\begin{figure}[H]
    \centering
    \includegraphics[scale=0.7]{ternary_unit_fitness.png}
    \caption[Ternary plot showing invasion capacity of patient squirrels under density dependence.]{Consider a population of $\beta = 0$ squirrels. We assume that ecological forces ensure the distribution of nuts is such
    that $\lambda_0(p_0, p_1, p_2) = 1$. Graphically, this means that the distribution is somewhere along the black line. If a mutant
squirrel with $\beta = 1$ arises in the population, then it will flourish in the nut rich environment provided by the $\beta = 0$ resident
population. As the $\beta = 1$ population expands, it places increasing pressure on the distribution of nuts so that the resident population
is driven to exctinction. }
    \label{fig:unit_fitness}
\end{figure}

Experimentally, it holds that for every cache level $\beta\in\N$, for each distribution $(p_0, p_1, p_2)$ satisfying $\lambda_\beta(p_0, p_1, p_2) = 1$,
the fitness of a slightly more patient mutant exceeds that of the resident population. That is, 
$$ \lambda_{\beta}(p_0, p_1, p_2) = 1 \implies  \lambda_{\beta+1}(p_0, p_1, p_2) > 1.$$
Therefore,
the process of $\beta = 0$ squirrels being invaded and overtaken by $\beta = 1$ squirrels repeats with $\beta = 2$ squirrels and later 
$\beta = 3, 4, 5\ldots$ squirrels. Therefore, an evolutionary approach yields the result that ever more patient mutants invade and fixate. In some
sense, there is an evolutionary ``race to the bottom,'' wherein resources are stretched ever more thinly across an increasing density
of squirrels. 

\section{The Evolution of Patience}
Squirrel banking is a highly constrained setting within which we analyze the adaptive pressures on patience.
Nevertheless, squirrel banking helps us to recognize some key elements in the evolution of patience beyond the model itself.

Consider the time dilemma faced by squirrels. The decision to bank to $\beta+1$ nuts rather than $\beta$ nuts before reproducing is,
above all else, a decision to delay reproduction in favour of producing more offspring over the entire lifetime. The induced smaller/sooner versus
larger/later decision squirrels make is therefore between total lifetime progeny and more quickly generating those progeny. Figure \ref{fig:lh_comparison}
allows us to compare the life histories of squirrels with identical growth rates. 



\begin{figure}[h]
    \centering
    \includegraphics[scale = 0.7]{lh_comparison.png}
    \caption[Life histories of patient and impulsive squirrels.]{We compare life histories to understand the patience tradeoff faced by squirrels. If the distribution of nuts
    is $(p_0, p_1, p_2) \approx (0.11, 0.47, 0.42)$, then the growth rates of $\beta = 0$ and $\beta = 1$ squirrels are
equal: $\lambda_0 \approx \lambda_1 \approx 1.1317$. The figure shows the surprising extent to which immediacy benefits population
growth. Whereas more patient squirrels produce considerably more progeny over the course of their lifetime, the three day period over which
impatient squirrels reproduce more is enough to make it so that the growth rate is the same between the two populations.}
    \label{fig:lh_comparison}
\end{figure}

Life history comparison raises an important point. Whereas lifetime progeny is a popular fitness measure in ecology \cite{stearns92}, it is
insufficient as a measure of reproductive success for our analysis of the evolution of patience. Instead, monomorphic population growth rate,
or the Lyapunov exponent $\lambda$, best describes reproductive success. 





\section{Conclusion}

Patience and intertemporal discounting are topics to which much attention has been paid, but about which there is still a surprising amount
left to uncover and understand. In particular, diverse fields like economics, psychology, and computer science each view consumption across
varying time horizons from fundamentally different vantage points and therefore come to starkly different conclusions. 
Evolutionary game theory is a relatively new set of tools which provides scaffolding for a structured and mathematical discussion of 
intergenerational developmental processes. 
We put forward the paradigme that patience is moulded by an evolutionary process, and therefore set out to use the tools from evolutionary
dynamics to reconcile the theory and practice of intertemporal reward valuation. 

We have herein presented an idealized model of squirrels foraging for and banking/consuming nuts. In this model, an individual's fitness
is assumed to be its average contribution to the gene pool over the course of its life. The setup and fitness constraints naturally lead to a
smaller/sooner versus larger/later dilemma: how many offspring to have, and when. Somewhat surprisingly,
our simple model is capable of generating a maximal diversity of outcomes: every level of patience permissible within the model framework 
optimizes fitness for some distribution of nuts. We further find evidence that a trait substitution sequence leads to extreme patience,
in which more patient squirrels invade resident populations forever into the future. 

While interesting in its own right, the model presented here is limited in its capacity to explain the evolutionary origins of intertemporal
discounting. Whereas squirrel banking touches on how delayed rewards are
valued less than immediate rewards in an evolutionary framework, we are unable to quantitatively evaluate the relative value of goods received
at different points in time. A nut today is worth more than a nut tomorrow for any given squirrel, but it is unclear exactly by how much. More 
general models are likely necessary to analytically understand the evolutionary pressures on consumption across time horizons. 

The present model is also perhaps overly simplistic. 
In practice, squirrels often bank nuts in preparation for winter
when nuts are not readily available. 
One could extend squirrel banking to account for seasonal variation in the
availability of nuts in conjunction with density dependence. On a practical level, the analytic intractibility of the current
setup leads us to believe further complications will make agent based simulations the only reasonable way to move forward toward an understanding  
of evolutionarily stable patience in squirrel banking. 

Despite its limitations and unrealistic elements, squirrel banking is a helpful step toward understanding the evolutionary dynamics of patience. 
In particular, the mechanistic setup of the model makes it clear that contribution to gene pool (as measured by unit time growth rate) is the most
reasonable measure of fitness, rather than lifetime reproductive output. Furthermore, we see that reproductive dilemmas (when do I reproduce?) 
are the most direct evolutionary take on classical smaller/sooner versus larger/later dilemmas. Overall, we hope that this article is the start
of a trend of applying the tools of evolutionary dynamics to problems not yet considered using those tools. 



\newpage
\bibliographystyle{alpha}
\bibliography{research}{}





\end{document}



