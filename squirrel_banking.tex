\documentclass[titlepage, hidelinks, 12pt]{article}


%for custom page numbering:
\usepackage{fancyhdr}

\usepackage{lipsum}
\usepackage{hyperref}
\usepackage{palatino}
\usepackage{tikz}
\usepackage{chngcntr}
\counterwithin{figure}{section}
%\usepackage{breqn} % useful for breaking equations across multiple lines automatically. Breaks everything.


\usepackage{setspace}
%\usepackage{indentfirst} %tex default is no indent on first paragraph after heading
\usepackage{url}
\usepackage{amsmath, amssymb, amsfonts, amsthm}
\usepackage{float}
\usepackage{subfig}
\usepackage{graphicx}
\usepackage{environ, enumerate}
%\usepackage{mathbbol}
%\DeclareSymbolFontAlphabet{\amsmathbb}{AMSb}
\graphicspath{ {images/} }
\providecommand{\keywords}[1]{\textbf{\textit{Keywords---}} #1} 
\usepackage[format=plain,
            labelfont={bf, it},
            textfont=it]{caption}





%%%%%%%%%
% indentation
%%%%%%%%%

\setlength\parindent{0pt}
\setlength{\parskip}{\baselineskip}

\setlength{\voffset}{-1cm}
\setlength{\textwidth}{17cm}
\addtolength{\textheight}{2cm}
\setlength{\footskip}{1cm}
\addtolength{\oddsidemargin}{-2cm}
\addtolength{\evensidemargin}{-2cm}

\widowpenalty10000
\clubpenalty10000



%%%%%%%%%
% page numbering and logo
%%%%%%%%%

\pagestyle{fancy}            

\fancyhead[L]{}
\fancyhead[C, R]{}
\fancyfoot[L]{}
\fancyfoot[C]{\thepage}
\fancyfoot[R]{\includegraphics[scale=.35]{sb_logo_black.png}}
\renewcommand{\headrulewidth}{0pt}
\renewcommand{\footrulewidth}{0pt}


%Def, Lemma, Theorem, Corollary environment
\theoremstyle{plain}
\newtheorem{theorem}{Theorem}[section]
\newtheorem{corollary}[theorem]{Corollary}
\newtheorem{lemma}[theorem]{Lemma}
\newtheorem{proposition}[theorem]{Proposition}
\newtheorem{conjecture}[theorem]{Conjecture}
%\newtheorem*{proof}{Proof}
\theoremstyle{remark}
\newtheorem*{remark}{Remark}
\newtheorem*{example}{Example}
\theoremstyle{definition}
\newtheorem{definition}[theorem]{Definition}

%New commands
\newcommand{\Q}{\mathbb{Q}}
\newcommand{\Z}{\mathbb{Z}}
\newcommand{\N}{\mathbb{N}}
\newcommand{\R}{\mathbb{R}}
\newcommand{\T}{\mathcal{T}}
\newcommand{\E}{\mathbb{E}}
\newcommand{\betahat}{\hat{\beta}}


%New math operators
\DeclareMathOperator{\ringchar}{char}
\DeclareMathOperator{\diag}{diag}
\DeclareMathOperator{\disc}{disc}
\DeclareMathOperator*{\argmax}{arg\,max}
%\DeclareMathOperator{\exp}{exp}
\renewcommand\d[1]{\:\textrm{d}#1}
\newcommand*\diff{\mathop{\!\mathrm{d}}}
\doublespacing
\begin{document}
\begin{singlespace}
\begin{titlepage}
   \begin{center}
       \vspace*{1cm}
\Huge
       \textbf{Squirrel banking} \\
\Large
       \vspace{0.5cm}
       Evolutionarily optimal caching. 
      
            
       \vspace{4.1cm}
       \includegraphics[scale = 1.4]{sb_logo_black.png} \\
       \vspace*{\fill}
       \textbf{Eric Rozon} \\

            
       Department of Mathematics\\
       University of British Columbia
            
   \end{center}
\end{titlepage}





\begin{abstract}

    First read, last written. 
    \bigskip

    \begin{keywords}
    Some, key, words. 
    \end{keywords}

\end{abstract}

\newpage

\tableofcontents

\newpage

%\section*{Acknowledgements}
%\addcontentsline{toc}{section}{Acknowledgement}
%My most sincere thanks are owed to Monica Nevins, without the support and guidance of whom none
%of this project would have been possible. Monica, it was a joy and a privilege to work so
%closely with you this semester, and I hope you are aware that my interest in mathematical research
%is due in no small part to your enthusiasm. I would also like to thank Professor Rafal Kulik for
%allowing me the opportunity to present part of this project during the University of Ottawa's final
%math club meeting of 2018. Finally, if you are reading this report, then thank you. While the project
%was stimulating enough in its own right that I do not feel it must be read by others to have been
%worthwhile, it is also fun to think that someone else may find some joy or usefulness in my work.
%\newpage
%\input{texfiles/outline}
%\newpage
\end{singlespace}





\section{Introduction}
We are often told that patience is a virtue. At the same times, the early bird gets the worm. 
While the sentiments expressed are in direct opposition, common sense dictates that each paradigme is of value. 
When making a decision between immediate and delayed gratification, the reconciliation of patiece and hastiness
may be translated into a question of how one ought to value time versus material gain. 

The theory of delay discounting has strong roots in economic theory \cite{sozou98, fedus19}. Models of discounting seek to axiomatize rational
behaviour and prescribe optimal preferences (patience or hastiness) facing decisions across different time horizons \cite{mazur97, ainslie75, green81}. 
Nevertheless, popular theories of discounting have severe limitations. In particular, experiments involving both human and non-human animals 
consistently find behaviour which is both qualitatively and quantitatively different from that predicted by standard 
models \cite{maia09, green04, frederick02}.

Axiomatic approaches to understanding decisions between patience and impulsiveness are insufficient. Instead, in this article we consider
an evolutionary approach to understanding patience. The brain (both human and non-human) often makes decisions that we do not fully consider
or rationalize; such brain functions are governed by what has been coined the adaptive unconscious \cite{wegner02}. In an evolutionary time scale,
those brains which make quick decisions which are favourable for gene proliferation are selected for and become the dominant brain type in a population.
In particular, we apply this reasoning to patience: those individuals whose patience is most adaptive will fixate in a population and drive all other
types to extinction.

In this article, we put forward an idealized model of squirrels foraging and banking nuts. Squirrels may either bank their nuts to guard against
starvation, or else consume additional nuts to produce offpsring. Ecologically, the squirrels in our model face a survival versus reproduction
tradeoff. From the perspective of patience, this translates into a question of fewer offspring more immediately (smaller and sooner benefit) 
or reproduction over a longer time span resulting in more offspring but later in life (larger and later benefit). By considering squirrels with
varying levels of patience (as measured by cache size before reproduction), we address the question: How much patience is evolutionarily optimal?


\section{Model}
Suppose a squirrel forrages daily for nuts. Each day, the squirrel finds either $0, 1,$ or $2$ nuts, with probabilities $p_0, p_1$, and $p_2$
respectively. If the squirrel consumes $0$ nuts, it dies. By consuming a single nut, the squirrel keeps itself alive. Finally, should the squirrel
consume $2$ nuts, it both keeps itself alive and produces an offspring. The offspring is subject to a one day period of parental care during
which it survives if and only if its parent survives. 

The life of a single squirrel is a stochastic process. The squirrel's survival and reproductive events are determined by the random process of
finding nuts. In a large (or infinite) population of squirrels, we assume nut findings are independent (both between individuals and day to day),
so that we may interpret that a fraction $p_n$ find $n$ nuts each day (for $n=0,1,2$). We consider the question of population growth: for what distributions 
$(p_0, p_1, p_2)$ will the squirrel population die out, remain constant, and grow exponentially? The introduction of a demographic matrix
helps to answer this question. Squirrels can be divided into two categories:
\begin{enumerate}
    \item Squirrels not providing parental care; and
    \item squirrels providing parental care.
\end{enumerate}

Transitions between the two states are governed by linear probabilities, so that the matrix $D_0$ encapsulates
all information necessary to understand the dynamics of the population of squirrels.
\begin{equation}
D_0 =
\begin{pmatrix}
    p_1 & 2(p_1 + p_2) \\ p_2 & 0
\end{pmatrix}
    \label{eqn:D0}
\end{equation}
We may codify an initial population of squirrels where no individual is providing parental care by the state vector $\left[ 1, 0 \right]^\intercal$. 
On day $t\ge 0$, the population will be in state $D_0^t\cdot \left[1, 0\right]^\intercal$. Standard results from the study of structured populations
imply (subject to minor diagonalizability requirements) that the largest positive eigenvalue $\lambda_0$ of $D_0$ is the growth 
rate of the population of squirrels. The population converges to a stable proportion in each class, and then whether $\lambda_0\gtreqless 1$ 
detetmines if the population is growing, constant, or decaying exponentially. Throughout, we refer to growth rate and fitness interchangeably. 

One objective for this model is to understand the evolution of patience. To the end of modelling differing levels of patience, we allow
squirrels to bank nuts to guard against starvation.
Rather than using all additional nuts to produce offspring immediately, a more patient population first caches an additional nut to guard against starvation
and later uses additional nuts to produce offspring. Intuitively, therefore, patient squirrels will delay reproduction but on average live
longer and so produce more offspring. A population of patient squirrels can be divided into three disjoint categories:
\begin{enumerate}
    \item Squirrels having cached $0$ nuts;
    \item squirrels having cached $1$ nut and not providing parental care; and
    \item squirrels providing parental care.
\end{enumerate}
Once more, we can encode all of the population dynamics of a mutant ``saver'' squirrel in a matrix $D_1$. 
\begin{equation}
D_1 =
\begin{pmatrix}
    p_1 & p_0 & 1+p_0 \\ p_2 & p_1 & 2(p_1 + p_2) \\ 0 & p_2 & 0
\end{pmatrix}
    \label{eqn:D1}
\end{equation}
As with $D_0$, the largest positive eigenvalue $\lambda_1$ of $D_1$ is the stable population growth rate, and whether $\lambda_0\gtreqless 1$ determines
the qualitative nature of the population dynamics (growth, constant size, or shrinking exponentially).

More generally, we may allow for squirrels to cache up to any level $\beta\in\left\{ 0, 1, 2, \ldots \right\}$ before reproducing. A population with cache
level $\beta$ can be divided into $\beta+2$ categories:
\begin{enumerate}
    \item Squirrels having cached $0$ nuts;
    \item squirrels having cached $1$ nut;
    \item \dots
    \item squirrels having cached $\beta$ nuts and not providing parental care; and
    \item squirrels providing parental care. 
\end{enumerate}
Now the population dynamics of squirrels banking up to $\beta$ nuts is given by the matrix $D_\beta$. As previously, the greatest positive eigenvalue
$\lambda_\beta$ of $D_\beta$ is the fitness of squirrels banking up to $\beta$ nuts before reproducing. 
\begin{equation}
D_\beta =
\begin{pmatrix}
    p_1 & p_0 & 0 & \cdots & 0 & 1 \\
    p_2 & p_1 & p_0 & \cdots & 0 & 0 \\
    0 & p_2 & p_1 & \cdots & 0 & 0 \\
    \vdots & \vdots & \vdots & \ddots & \vdots & \vdots \\
    0 & 0 & 0 & \cdots & p_0 & p_0 \\ 
    0 & 0 & 0 & \cdots & p_1 & p_1 + p_2 \\ 
    0 & 0 & 0 & \cdots & p_2 & 0 \\ 

\end{pmatrix}
    \label{eqn:Dbeta}
\end{equation}

Towards understanding adaptive pressures on patience, we analyse the relationship between patience and fitness. An increase in patience has two
opposing effects. On the one hand, banking more nuts results in higher survival probability and therefore increases like expectancy and lifetime
reproductive output. On the other hand, higher patience results in offspring being produced later in life which tends to slow the rate of population
growth. Therefore, for each distribution of nuts $(p_0, p_1, p_2)$, we ask: what is the banking level $\beta$ which maximizes fitness $\lambda$? 
Mathematically, we seek to compute 
\begin{equation}
    \beta^*(p_0, p_1, p_2) := \argmax_{\beta\in\N} \lambda_\beta.
    \label{eqn:betastar}
\end{equation}

The problem described in equation \ref{eqn:betastar} is analytically intractible. We get some intuitive understanding of the nature of the problem
by numerical approximations. For instance, for $(p_0, p_1, p_2) = (0.3125, 0.3125, 0.375)$, we compute:
\begin{table}[H]
    \centering
\begin{tabular}{l|l}
$\beta$ & $\lambda_\beta$    \\ \hline
0       & 0.891123501019053  \\
1       & 1.0042807386908956 \\
2       & 1.0109688024833265 \\
3       & 1.0119993543876564 \\
4       & 1.0116325207441923 \\
5       & 1.010911192617546  \\
$\vdots$ & $\vdots$
\end{tabular}
\end{table}
From numerical approximations, the trend is that $\lambda_{\beta}$ increases with $\beta$ until it reaches a peak, at which it is maximized. 
From the table, we read off that $\beta^*(0.3125, 0.3125, 0.375) = 3$. 

More generally, our numerical work indicates that the optimal cache level $\beta^*$ increases as the distribution of nuts becomes less bountiful.
That is, squirrels foraging in an environment where nuts are plentiful have maximal fitness when they exhibit no patience and use all available
resources for reproduction. If nuts are less plentiful, however, squirrels are better served to be patient and delay reproduction in favour of
higher survival probability. A pleasing result of our model setup is that for every cache level $\beta$, there exists some distribution of nuts
$(p_0, p_1, p_2)$ for which $\beta^*(p_0, p_1, p_2) = \beta$.


\begin{figure}[H]
    \centering
    \includegraphics[scale = 0.7]{ternary_optimal_beta.png}
    \caption{The simplex of points $(p_0, p_1, p_2)$ such that $p_0 + p_1 + p_2 =1$ describes the space of all possible nut distributions.
        The simplex can be projected onto two dimensional space as a ternary diagram. Each point is a distribution of nuts $(p_0, p_1, p_2)$,
        and 
    the colouring of each point indicates the cache level resulting in maximal fitness. When nuts are plentiful (upper right of the diagram),
squirrels maximize fitness by banking $0$ nuts. If nuts are less available, meaning the distribution shifts toward the bottom left, 
then banking more nuts is favourable. As $p_2\to p_0$, $\beta^*\to\infty$. For visualization purposes, we stop at $\beta^* = 6$. 
If $p_0 > p_2$, meaning a squirrel is more likely to find $0$ nuts than $2$ nuts, then fitness $\lambda_\beta < 1$
for every finite $\beta$. Infinite patience is required to maintain constant population size.}
    \label{fig:optimal_beta}
\end{figure}



\subsection{Discussion and Extension}
The present setup assumes the distribution of nuts $(p_0, p_1, p_2)$ is constant in time. The results shown in Figure \ref{fig:optimal_beta}
are therfore the dominant patience phenotype for each constant distribution $(p_0, p_1, p_2).$ However, it is unreasonable to suppose that
exponential growth takes place indefinitely. 
A more likely alternative is that resources
are density dependent. Without specifying the mechanism via which population density determines the distribution of nuts, we abstractly
posit that some such mechanism exists and ensures that populations neither grow nor decay exponentially, and rather remain at a constant size.


Let us suppose that a population of $\beta = 0$ squirrels is in equilibrium. 
Ecological forces have driven the distribution of nuts to some $(p_0, p_1, p_2)$ at which $\lambda_0 =1$,
meaning the squirrel population is of constant size. Spontaneously, a mutant squirrel with cache level $\beta = 1$ is born. Computing numerically,
we see that for all distributions $(p_0,p_1,p_2)$ satisfying $\lambda_0(p_0, p_1, p_2) = 1$, $\lambda_1(p_0,p_1,p_2)>1$. That is, the fitness
of the mutant (patient) squirrel is greater than that of the resident population. It follows that the mutant population will grow, putting additional
pressur on resources. The mutant population therefore profits at the expense of the residents, until the $\beta = 0$ squirrels have all died off
and the $\beta = 1$ squirrels have driven nut distribution to some $(p_0', p_1', p_2')$ satisfying $\lambda_1(p_0', p_1', p_2') = 1$. 

\begin{figure}[H]
    \centering
    \includegraphics[scale=0.7]{ternary_unit_fitness.png}
    \caption{Consider a population of $\beta = 0$ squirrels. We assume that ecological forces ensure the distribution of nuts is such
    that $\lambda_0(p_0, p_1, p_2) = 1$. Graphically, this means that the distribution is somewhere along the black line. If a mutant
squirrel with $\beta = 1$ arises in the population, then it will flourish in the nut rich environment provided by the $\beta = 0$ resident
population. As the $\beta = 1$ population expands, it places increasing pressure on the distribution of nuts so that the resident population
is driven to exctinction. }
    \label{fig:unit_fitness}
\end{figure}

Experimentally, it holds that for every cache level $\beta\in\N$, for each distribution $(p_0, p_1, p_2)$ satisfying $\lambda_\beta(p_0, p_1, p_2) = 1$,
the fitness of a slightly more patient mutant exceeds that of the resident population. That is, $\lambda_{\beta+1}(p_0, p_1, p_2) > 1$. Therefore,
the process of $\beta = 0$ squirrels being invaded and overtaken by $\beta = 1$ squirrels repeats with $\beta = 2$ squirrels and later 
$\beta = 3, 4, 5\ldots$ squirrels. Therefore, an evolutionary approach yields the result that ever more patient mutants invade and fixate. In some
sense, there is an evolutionary ``race to the bottom,'' wherein resources are stretched ever more thinly across an increasing density
of squirrels. 






\section{Conclusion}

Patience and intertemporal discounting are topics to which much attention has been paid, but about which there is still a surprising amount
left to uncover and understand. In particular, diverse fields like economics, psychology, and computer science each view consumption across
varying time horizons from fundamentally different vantage points and therefore come to starkly different conclusions. 
Evolutionary game theory is a relatively new set of tools which provides scaffolding for a structured and mathematical discussion of 
intergenerational developmental processes. 
We put forward the paradigme that patience is moulded by an evolutionary process, and therefore set out to use the tools from evolutionary
dynamics to reconcile the theory and practice of intertemporal reward valuation. 

We have herein presented an idealized model of squirrels foraging for and banking/consuming nuts. In this model, an individual's fitness
is assumed to be its average contribution to the gene pool over the course of its life. The setup and fitness constraints naturally lead to a
smaller/sooner versus larger/later dilemma: how many offspring to have, and when. Somewhat surprisingly,
our simple model is capable of generating a maximal diversity of outcomes: every level of patience permissible within the model framework 
optimizes fitness for some distribution of nuts. We further find evidence that a trait substitution sequence leads to extreme patience,
in which more patience squirrels invade resident populations forever into the future. 

While interesting in its own right, the model presented here is limited in its capacity to explain the evolutionary origins of intertemporal
discounting. Whereas squirrel banking touches on how delayed rewards are
valued less than immediate rewards in an evolutionary framework, we are unable to quantitatively evaluate the relative value of goods received
at different points in time. A nut today is worth more than a nut tomorrow for any given squirrel, but it is unclear exactly by how much. More 
general models are likely necessary to analytically understand the evolutionary pressures on consumption across time horizons. 

The present model is also perhaps overly simplistic. 
In practice, squirrels often bank nuts in preparation for winter
when nuts are not readily available. 
One could extend squirrel banking to account for seasonal variation in the
availability of nuts in conjunction with density dependence. On a practical level, the analytic intractibility of the current
setup leads us to believe further complications will make agent based simulations the only reasonable way to move forward toward an understanding  
of evolutionarily stable patience in squirrel banking. 

Despite its limitations and unrealistic elements, squirrel banking is a helpful step toward understanding the evolutionary dynamics of patience. 
In particular, the mechanistic setup of the model makes it clear that contribution to gene pool (as measured by unit time growth rate) is the most
reasonable measure of fitness, rather than lifetime reproductive output. Furthermore, we see that reproductive dilemmas (when do I reproduce?) 
are the most direct evolutionary take on classical smaller/sooner versus larger/later dilemmas. Overall, we hope that this article is the start
of a trend of applying the tools of evolutionary dynamics to problems not yet considered using those tools. 



\newpage
\bibliographystyle{alpha}
\bibliography{research}{}





\end{document}



